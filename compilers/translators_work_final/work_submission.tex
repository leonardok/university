\documentclass[12pt]{article}

\usepackage{sbc-template}
\usepackage{graphicx,url}


\usepackage{listings}
\usepackage{color}
\usepackage{textcomp}
%\definecolor{listinggray}{gray}{0.9}
%\definecolor{lbcolor}{rgb}{0.9,0.9,0.9}
\lstset{
	%backgroundcolor=\color{lbcolor},
	tabsize=4,
	rulecolor=,
	basicstyle=\scriptsize,
	upquote=true,
	aboveskip={1.5\baselineskip},
	columns=fixed,
	showstringspaces=false,
	extendedchars=true,
	breaklines=true,
	prebreak = \raisebox{0ex}[0ex][0ex]{\ensuremath{\hookleftarrow}},
	%frame=single,
	showtabs=false,
	showspaces=false,
	showstringspaces=false,
	identifierstyle=\ttfamily,
	keywordstyle=\color[rgb]{0,0,1},
	commentstyle=\color[rgb]{0.133,0.545,0.133},
	stringstyle=\color[rgb]{0.627,0.126,0.941}
}

\usepackage[brazil]{babel}   
\usepackage[latin1]{inputenc}  

     
\sloppy

\title{Instructions for Authors of SBC Conferences\\ Papers and Abstracts}

\author{Aristeu F. de Oliveira Junior\inst{1}, Leonardo R. Korndorfer\inst{1}, Thais Dyana M. Souza\inst{1} }


\address{
  UNISINOS\\ 
  Av. Unisinos, 950 -- B. Cristo Rei, CEP 93.022-000 -- S�o Leopoldo (RS)
  \email{\{aristeuoliveirajunior,leokorndorfer,thaisdms\}@gmail.com}
}

\begin{document} 

\maketitle

\begin{abstract}
	Ruby is an interpreted language that has a growing user base. It is used for a wide variety of applications such as web applications, server administration scripts, desktop applications and so on. Ruby growth is attributed to the nice language syntax, short learning curve and other subjective attributes like a friendly community.
This work intend to design and describe the implementation proposal for a ruby to C translator, using the thechniques presented in class, that aim to make a start point for fast code to be run where you need more performance using the ruby features and the fast production time.
\end{abstract}
     
\begin{resumo} 
	Ruby � uma linguagem interpretada que tem uma base de usu�rios em expans�o. A linguagem � usada para uma grande variedade de aplica��es como aplicativos para a web, script para administra��o de servidores, aplicativos para desktop e outros. O crescimento do ruby � atribu�do a boa sintaxe da linguagem, pequena curva de aprendizado e outros pontos subjetivos, como uma comunidade amig�vel.
Este trabalho tem como objetivo o design e descri��o da proposta de implementa��o para um tradutor de ruby para C, usando as t�cnicas apresentadas em aula, que objetiva um ponto de partida para que c�digo mais r�pido possa ser executado onde se necessita mais performance, mas usando as qualidades e rapidez de produ��o de ruby.
\end{resumo}


\section{Linguagem Origem}

A linguagem origem para o nosso trabalho sera ruby, por se tratar de uma 
linguagem nova e com bastante ascencao. Outro fator de motivacao e a nao
existencia de compiladores para a linguagem, que e interpretada, o que
leva o trabalho a um objetivo maior de talvez produzir um ponto de partida
para um compilador para a linguagem, podendo contribuir com a comunidade 
open source.


\section{Linguagem Alvo}

A liguagem alvo para o nosso tradutor - linguagem de output - sera C, por ser
uma linguagem que possui bons compiladores e rapidos executaveis. A linguagem
tambem e dominada pelo grupo.

Limitacoes
biliotecas
acesso a atributos e funcoes
passagem de codigo via parametro


\section{Exemplo de Entrada e Saida Esperadas}

\subsection{Entrada}
\begin{lstlisting}[language=ruby]
class Person
  def initialize(fname, lname)
   @fname = fname
   @lname = lname
  end
end
person = Person.new("Augustus","Bondi")
print person
\end{lstlisting}

\subsection{Saida}
\begin{lstlisting}[language=c]
/* include section */

struct person_s {
	char *fname;
	char *lname;
};
typedef struct person_s person_t;


/* function prototypes */
void initialize(char *, char *);

void initialize(char *fname, char *lname)
{
	person.fname = malloc(sizeof(char) * (strlen(fname) + 1));
	strcpy(person.fname, fname);
	strcat(person.fname, '\0');
	
	person.lname = malloc(sizeof(char) * (strlen(lname) + 1));
	strcpy(person.lname, lname);
	strcat(person.lname, '\0');
}

void print_person(person_t person)
{
	printf("%s %s\n", fname, lname);
}

int main(int argc, char **argv)
{
	person_t *person = malloc(sizeof(person));;
	initialize("Augustus","Bondi");

	print_person(person);

	return 0;
}
\end{lstlisting}


\section{Ferramenta a ser Ultilizada}

Bison

\section{References}

Bibliographic references must be unambiguous and uniform.  We recommend giving
the author names references in brackets, e.g. \cite{knuth:84},
\cite{boulic:91}, and \cite{smith:99}.

The references must be listed using 12 point font size, with 6 points of space
before each reference. The first line of each reference should not be
indented, while the subsequent should be indented by 0.5 cm.

\bibliographystyle{sbc}
\bibliography{sbc-template}

\end{document}
